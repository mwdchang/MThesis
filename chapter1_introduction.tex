%%%%%%%%%%%%%%%%%%%%%%%%%%%%%%%%%%%%%%%%%%%%%%%%%%%%%%%%%%%%%%%%%%%%%%%%%%%%%%%%
% This chapter summarizes the thesis work 
% 
% - Some kind of graphics would be nice? showing taking text to final image?
%%%%%%%%%%%%%%%%%%%%%%%%%%%%%%%%%%%%%%%%%%%%%%%%%%%%%%%%%%%%%%%%%%%%%%%%%%%%%%%%
\chapter{Introduction}

Many types of text data are generated and archived on a daily basis, such as
diaries, blogs and incidents reports. These documents contain a wealth of
information that can be extracted and analyzed. For example, various statistical
measures, structures of the documents and word semantics. The field of
Information Visualization (InfoVis) deals  with the visual representation of
these documents to amplify human cognition~\cite{Card1999}, in particular, to
summarise and explore the dataset. While there are a broad range of InfoVis
techniques for text documents, few visualization applications leverage the
fact that text documents can have real-world dimensions such as words referring
to physical subjects or inherent spatial structures: rather then showingi
these dimensions in a recognizable form, these applications present abstract
renderings that are out of context.

Consider a review article of a bicycle, it contains words
referring to bicycle parts that have real-world counterparts, there are also
inherent spatial relations among different bicycle parts. Now, consider the
popular technique of word-clouds (\eg, ~\cite{VIE2009a}): the visualization
arranges words with respect to frequency, including the physical words which
describe the bicycle components. However, the rendering is rather arbitrary, the
final product does not resemble a bicycle and the placement of the words do not
suggest any spatial relationships.

In this work, we introduce descriptive non-photorealistic rendering, a hybrid
approach for text analytics that leverages the real-world dimensions in text
document. This process creates information rich visuals such that the
significant parts appear to ``popout'' to the viewers, while retaining shapes
and forms that are easily recognizable and in context with real-world
expectations.

%In this work, we introduce descriptive non-photorealistic rendering, a hybrid
%approach for text analytics which reveals both abstract and spatial semantics in
%a single visualization. This process creates information rich visuals such that
%significant parts of the model appear to �popout� to the viewers, while
%retaining their easily recognizable shapes and forms. 

%Consider the popular technique of word-clouds (\eg,
%~\cite{VIE2009a}) and using it to visualize a bicycle review: the document
%contains words referring to bicycle parts that have real-world manifestations,
%as well in inherent spatial relation among different parts. 


%Many types of text data are generated and archived on a daily basis, such
%as diaries, blogs and incident reports. These documents can be broadly
%categorized to have two types of attributes: abstract attributes that deal with
%word semantics that are intangible in nature; and concrete attributes where the
%words or relations among the words have physical manifestations. The field of Information
%Visualization (InfoVis) deals with the visual representation of these documents
%to amplify human cognition~\cite{Card1999}, in particular, to summarize and
%explore the dataset. While there are a broad range of InfoVis techniques for
%text documents, they often do not integrate the visualization of abstract and
%concrete attributes, and when they do, the visualization is presented in
%an abstract manner. Consider the popular technique of word cloud (\eg,
%\cite{VIE2009a}): the words contain concrete attributes such as physical
%objects, the size of the tags is linked to the word frequency (abstract
%attributes), but the rendering of the cloud visualization itself does not
%reflect the physical meaning of these words.




%While there are a broad range of InfoVis techniques for
%text documents, they often do not integrate the visualizations of abstract and
%concrete attributes. Consider the popular technique of word cloud (\eg, 
%\cite{VIE2009a}): the size of tags is linked to the word frequency, but the
%rendering of the cloud does not reflect the physical meaning of these words.  
 
 
% The amount of data available today is staggering. With advanced technologies,
% data are constantly been created and archived. Understanding large quantities of data is
% a difficult challenge in terms of the sheer volume and the time effort needed to
% sift through the data to find salient information. Information Visualization
% (InfoVis), is a field that deals with this data overload by using visual
% representations of data to amplify human cognition \cite{Card1999}. While there
% are many InfoVis techniques that deal with summarization and exploration of
% data, particularly in text format, these techniques are not always in context
% because of their abstract presentation. For example, consider the popular
% technique of TagCloud, the sizes of the tags are linked to the
% frequency, but the actual context and subject matter of the text is not taken
% into account. Many of these missing contexts are concrete ideas about objects
% and space relations that do not naturally lend themselves to be mapped against
% abstract representations.
 
%In this work, we introduce descriptive non-photorealistic rendering, a hybrid
%approach for text analytics which reveals both abstract and spatial semantics in
%a single visualization. This process creates information rich visuals such that
%significant parts of the model appear to �popout� to the viewers, while
%retaining their easily recognizable shapes and forms. 

  
   
\section{Motivation}
There are many text collections about physical things, and information
visualization generally summarizes these documents with an abstract
presentation. While these visualizations are easy to understand, they are
typically disconnected from the real world and have no physical resemblance  to
the subject matter in the text.
In another words, we do not seem to be taking advantage of existing familiarity
of the subject matter. Creating this link may have several benefits; first it is
immediately obvious upon viewing what the text documents are about; and second, 
it opens up a wide array of exploration opportunities because the physical/spatial dimension
becomes available. Many applications already exhibit these properties, though
not necessarily for text visualization. For example consider map applications
with a point-of-interests overlay, the map exposes the spatial attributes while
the point-of-interests provides the abstract semantics.

The second part of our motivation came from illustrations of objects found in
technical or medical materials, these illustrations are often drawn in a way 
to attract the viewer's attention to specific parts of the illustration. This 
stylistic approach, known as Non-Photorealistic Rendering (NPR) outshines 
traditional photographic pictures in its capability to carry a specific message 
to the viewers. For example, putting emphasis on sub sections of the image such 
that they \emph{pop-out} visually, thus making the region more salient while 
retaining enough visual information that the object is easily recognizable. This
type of technique is fairly common in the Scientific Visualization community
where volumetric data are often differentiated with colour and texture.
However, such techniques are less commonly used in InfoVis as most illustrations
do not depict physical objects. 

In this work, we want to explore ways of combining our two motivations together
into a single interactive visualization: a system that can be used to summarize
text documents pertaining to physical artifacts by means of NPR techniques. 


 
%Motivated by the lack of coverage in this area, we want to investigate how an
%integrated approach of using both InfoVis and NPR graphics can enhance visual
%exploration of data.

%We foresee several benefits for applying NPR techniques for InfoVis based
%systems, especially those systems that explores real world entities. Foremost,
%NPR based illustrations preserve a sense of realism, objects under NPR are
%recognizable to the viewers based on their past experiences, without the need to
%interpret text or abstract visual encoding. Location based patterns are also
%perceivable as they are positioned relative to their actual spatial locations.
%Together, we think this can enhance the user experiences in visually exploring
%the underlying data.

 
\section{Approach}
There are several challenges that need to be addressed to successfully merge
abstract text visualization with NPR illustrations: finding a way to formally
define the subject matter, encode the NPR graphics, and finally create
interactions to navigate and to explore the underlying dataset. 

For the first problem, we extracted from the text documents the physical noun
keywords that make up the subject matter. A meronomy (part-of) relationship is
used to link the noun entities together to from a hierarchy representing the physical parts of
the subject. We then segmented \threed models that represent the subject into
different geometric groups to match our hierarchy, creating a 1-to-1 mapping 
between the keyword ontology and the \threed model components.

Second, we defined the semantic relations that we want to visualize, which are
occurrence and co-occurrence relations of the physical entities in the text. We
created a mapping function that takes the entity relations as parameters and outputs a
stylized graphical effect which is applied to the geometric components, this is used to 
denote the level or strength of the semantic relations. The subject matter can 
then be reconstructed with the aforementioned hierarchy of parts, creating the 
main view of our visualization that showcases the entities with highly scored
relations.

Lastly, to enable exploration of data, we designed a set of widgets to perform
filter and drill-down operations. The widget are built to take advantage of the
exposed spatial dimensions, by operating over regions of space as well as over individual
entity elements.

We call our approach \emph{descriptive non-photorealistic rendering}, as we combined
the summarization of thousands of documents with the message-carrying nature of 
non-photorealistic illustrations.

To demonstrate our application in a realistic context, we applied our methods
to analyze a text corpus of vehicle complaint documents from the US National
Highway Traffic Safety Administration (NHTSA). We conducted a system evaluation 
study with 12 participants. Our goal is to asses if, and how a person can use
the visualization to facilitate his/her analytical tasks. 

%In particular we asses their ability to interpret
%the visualization and how they use the system to perform open-ended analytical
%tasks.

%In particular we assess their ability to accurately
%interpret the visualization and how they perform open-ended tasks.


%In particular we allow queries to be executed
%visually via a widget that embodies the lens metaphor. We then mapped simple
%touch gestures onto our user interface, to enable our visualization to run on a
%large display in a walk-up-and-use case scenario.
% There are three main problems that should be addressed here: how to handle the
% text document, how to visualize the document, and lastly how to interact with
% the visualization data.
% 
% For the first problem, our system analyzes collections of descriptive text,
% extracting entities that forms a \emph{part-of} relationship known as meronomy
% relationship. We then construct a scoring function that tells us how frequently
% entity occurs in the document, as well as their relative occurrence to each
% other. 
%  
% In the second problem, we built a virtual representation of the document
% entities. We start with a segmented \threed geometric model that is mapped onto
% our meronomy ontology. We then encode the scores above onto each segments, 
% using colour, transparency and other stylization styles to put emphasis on  an
% entity. We render the model as precisely as possible to retain the familiarity
% one would have with the entities in the real world.
%  
% Lastly, to enable exploration of data, we designed a set of widgets to perform
% filter and drill-down operations. In particular we allow queries to be executed
% visually via a widget that embodies the lens metaphor. We then mapped simple
% touch gestures onto our user interface, to enable our visualization to run on a
% large display in a walk-up-and-use case scenario.
% 
% We call our approach \emph{descriptive non-photorealistic rendering}, as we combined
% the summarization of thousands of documents with the message carrying nature of 
% non-photorealistic illustrations.
% 
% To demonstrate our application in a realistic context, we applied our methods
% to analyze a text corpus of vehicle complaint documents from the US National
% Highway Traffic Safety Administration (NHTSA). We conducted an user evaluation 
% study with 12 participants. In particular we assess their ability to accurately
% interpret the visualization and how they perform open-ended tasks.


\section{Contribution}
This work describes \emph{descriptive non-photorealistic rendering}, a
novel approach for visualizing text documents with both concrete and abstract
attributes. By combining the different semantics into a single view and using
non-photorealistic rendering techniques, we created an engaging visualization
that can be related to the real world.

% This work describes a novel approach for visualizing text documents with inherent spatial 
% attributes that we called \emph{descriptive non-photorealistic rendering}.
% By combining abstract semantics with non-photorealistic rendering techniques into a 
% single view,  we created an engaging visualization that can be related to the real world.

% This work describes a novel approach for conducting text analytic tasks.
% Combining together Non-Photorealistic Rendering,\threed rendering and
% Information Visualization techniques we created a prototype application capable
% of analyzing thousands of text documents pertaining to physical entities.
 

  
\section{Organization}
Chapter 2 provides a literature review that discusses related work and
research that served as our inspirations. Chapter 3 discusses the problem and
our solution, it also introduces our working dataset for the remainder of this
work. Chapters 4, 5 and 6 discuss our visualization designs. In Chapter 7 we
describe our system architecture and solutions to usability issues encountered
during implementation phase. Chapter 8 contains our evaluation and use case
scenarios. Finally, in Chapter 9 we summarize our contributions and discuss
avenues for future work.

% The remainder of the thesis is organized as follows: 
%  
% \noindent Chapter 2 is a literature review that discusses the relevant and
% related works.
%  
% \noindent Chapter 3 discusses the problem and our proposed solution. It also
% introduces a problem scenario which we used to frame our discussions for the remainder of this thesis.
%  
% \noindent Chapter 4 deals with document process, it describes the steps to tag
% and score the document text.
%  
% \noindent Chapter 5 describes our visual and interaction design. 
%  
% \noindent Chapter 6 looks at the visualization system's capabilities to support
% analytical tasks
%  
% \noindent In chapter 7 we describe our system architecture, in additional, we
% detail several non-trivial algorithms used in the implementation.
% 
% \noindent Chapter 8 introduces use case scenarios, along with our user
% study evaluation and results.
% 
% \noindent Finally, in Chapter 9 we summarize our contributions and discuss
% avenues for future work. 
  
 
  