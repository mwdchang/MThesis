\chapter{Introduction}
 With data constantly growing, understanding large quantities of text is a
 difficult challenge in terms of the sheer volume and lack of task-driven visual
 techniques by which text documents can be effectively summarized \cite{ROH2011a}.
 People often resort to summarization techniques to extract the main themes in a
 collection of text. Popular visual summarization techniques such as TagCloud
 and WordCloud (Cite) render frequently occurring words in an abstract array,
 often out of context and without considering the underlying semantics of the
 words, nor the overall subject matter. However, many words in a text have
 easily discovered relationships. One type of this relation is meronymy or the
 \emph{part-of} relationship. Using a meronym database, it is possible to
 construct a part-whole relationship. In this thesis, we focus on the
 visualization of meronyms that have implicit spatial relationship, that is,
 they have physical representations in the real world. 
 
 
 \section{Motivation and Approach}
 Illustrations in technical and medical textbooks are typically drawn in a
 manner that is closely tied to Non-Photorealistic Rendering (NPR). Unbounded by
 physical constraints, these illustrations takes on a plethora of rendering
 styles, putting emphasis on different subsections of the rendered scene. Thus,
 these illustrations are information-rich entities, not only do these
 subsections \emph{pop-out} to the viewers, they still retain their easily
 recognizable shape and forms. While NPR techniques are used prominently in the
 field of Scientific Visualization, we have seen few examples of its usage in
 Information Visualization, specifically the use of NPR techniques to augment
 text-analytic tasks.
 
 Here we investigate an integrated approach for analyzing text content. Our
 system analyzes collections of descriptive text, extracts mentions of parts of
 physical entities and computes an importance score. We then encode the
 resulting scores onto a virtual \threed representation of these entities,
 with the magnitude of the score determining the type of graphical
 embellishments. Finally, we create interactions to allow people to visually
 query and filter the data based on their pre-existing spatial awareness and
 familiarity of the subject.
 
 
 To demonstrate our application in a realistic scenario, we applied our methods
 to analyze a text corpus composed of over 800,000 vehicle complaint reports
 from the US National Highway Traffic Safety Administration (NHTSA). Our goal is
 to help consumers make better purchasing decisions.
 
 %\begin{figure}
 % \centering 
 % \includegraphics[width=\columnwidth]{test.jpg}
 % \caption{A subset of the component keywords hierarchy.  Words in square brackets are synonyms.  The components in this hierarchy were isolated in the \threed model of a car, and extracted from the text data and counted.}
 % \label{figure:meronym}
 %\end{figure}
  
 
 > Presenting visualization that promotes more collaboration ? > NPR Rendering techniques
 
 \section{Contribution}
 This work describes a novel approach for conducting text analytic tasks.
 Combining together non-photorealistic rendering,\threed rendering and
 Information Visualization techniques we created a prototype application capable
 of analyzing thousands of text documents pertaining to physical entities.
 

 
 \section{Organization}
 The remainder of the these is organized as follows:
 
 \noindent Chapter 2 discusses related works from research literature. 
 
 \noindent Chapter 3 discusses the problem scenario in depth, it describes how
 we came up with our goals and requirement for design.
 
 \noindent Chapter 4 presents our visual design and our justifications. 
 
 \noindent Chapter 5 discusses high level interactions, in particular we
 discuss typical usage scenarios of the visualization system
 
 \noindent Chapter 6 discusses our user study design and study results. 
 
 \noindent Finally, in Chapter 7 we summarize our contributions and discuss
 avenues for future work. 
  
 
  