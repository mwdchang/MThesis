\chapter{Introduction}
The amount of data available today is stagger. With advanced technologies, data
are constantly been created and archived. Understand large quantities of data is
a difficult challenge in terms of sheer volume and the time effort needed to
sift through the data to find salient information. Information Visualization
(InfoVis), is a field that deals with this data overload by using visual
representations of data to amplify human cognition \cite{Card1999}. While there
are many InfoVis techniques that deals with summarization and exploration of
data, particularly in text format, these techniques are not always in context
because of their abstract presentation. For example, consider the popular
technique of TagCloud, the size of the tags are semantically linked to the
frequency, but the actual context and subject matter of the text is not taken
into account. Many of these missing context are concrete ideas about objects and
space relations that do not naturally lend themselves to be mapped against
abstract representations.
 
In this work, we explore an alternative approach for summarizing text documents
by preserving the physical entities in the text as they would appear in the real
world. We realize these entities as \threed geometries and encode additional
semantics as graphical effects. We then design interactions to navigate and
explore the \threed visualization space.

 %With data constantly growing, understanding large quantities of text is a
 %difficult challenge in terms of the sheer volume and lack of task-driven visual
 %techniques by which text documents can be effectively summarized \cite{ROH2011a}.
 %People often resort to summarization techniques to extract the main themes in a
 %collection of text. Popular visual summarization techniques such as TagCloud
 %and WordCloud (Cite) render frequently occurring words in an abstract array,
 %often out of context and without considering the underlying semantics of the
 %words, nor the overall subject matter. However, many words in a text have
 %easily discovered relationships. One type of this relation is meronymy or the
 %\emph{part-of} relationship. Using a meronym database, it is possible to
 %construct a part-whole relationship. In this thesis, we focus on the
 %visualization of meronyms that have implicit spatial relationship, that is,
 %they have physical representations in the real world. 
 
 \section{Motivation}
 Illustrations found in technical or medical materials are often drawn in a way
 to attract the viewer's attention to specific parts of the illustration. This
 stylistic approach, known as Non-Photorealistic Rendering (NPR) outshines
 traditional photographic pictures in its capability to carry a specific
 message to the viewers, for example, putting emphasis on sub sections of the
 image such that they \emph{pop-out} to the viewers. While the idea of using 
 NPR is fairly common in the field of Scientific Visualization; from the 
 InfoVis side, we have seen few example usages of NPR techniques to  highlight
 interesting data.
 
 Motivated by the lack of coverage in this area, we want to investigate how an
 integrated approach of using both InfoVis and NPR graphics can enhance visual
 exploration of data.
 
 \section{Approach}
 There are three main problems that should be addressed here: how to handle the
 text document, how to visualize the document, and lastly how to interact with
 the visualization data.
 
 For the first problem, our system analyzes collections of descriptive text,
 extracting entities that forms a \emph{part-of} relationship known as meronomy
 relationship. We then construct a scoring function that tells us how frequently
 entity occurs in the document, as well as their relative occurrence to each
 other. 
 
 In the second problem, we built a virtual representation of the document
 entities. We start with a segmented \threed geometric model that is mapped onto
 our meronomy ontology. We then encode the scores above onto each segments, 
 using colour, transparency and other stylization styles to put emphasis on  an
 entity. We render the model as precisely as possible to retain the familiarity
 one would have with the entities in the real world.
 
 Lastly, to enable exploration of data, we designed a set of widgets to perform
 filter and drill-down operations. In particular we allow queries to be executed
 visually via a widget that embodies the lens metaphor. We then mapped simple
 touch gestures onto our user interface, to enable our visualization to run on a
 large display in a walk-up-and-use case scenario.
 
 To demonstrate our application in a realistic context, we applied our methods
 to analyze a text corpus composed of over 800,000 vehicle complaint reports
 from the US National Highway Traffic Safety Administration (NHTSA). Our goal is
 to help consumers make better purchasing decisions.
 
 
 \section{Contribution}
 This work describes a novel approach for conducting text analytic tasks.
 Combining together non-photorealistic rendering,\threed rendering and
 Information Visualization techniques we created a prototype application capable
 of analyzing thousands of text documents pertaining to physical entities.
 

 
 \section{Organization}
 The remainder of the these is organized as follows:
 
 \noindent Chapter 2 discusses related researches and other inspirational works
 from research literature.
 
 \noindent Chapter 3 discusses the problem scenario in depth, it describes how
 we came up with our goals and requirement for design.
 
 \noindent Chapter 4 presents our design and justifications. This chapter is
 broken down into 3 main subsections, data handling, visual design and
 visualization interactions.
 
 \noindent Chapter 5 discusses in our architecture and implementation. In
 addition we address some non-trivial problems that we came across during the
 development of this visualization.
 
 \noindent Chapter 6 discusses our user study design and study results. 
 
 \noindent Finally, in Chapter 7 we summarize our contributions and discuss
 avenues for future work. 
  
 
  