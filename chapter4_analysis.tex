%%%%%%%%%%%%%%%%%%%%%%%%%%%%%%%%%%%%%%%%%%%%%%%%%%%%%%%%%%%%%%%%%%%%%%%%%%%%%%%%
% This chapter describes the analytics tools and views
%
% - Should rewrite comparison section, see pacivicVIS paper
%%%%%%%%%%%%%%%%%%%%%%%%%%%%%%%%%%%%%%%%%%%%%%%%%%%%%%%%%%%%%%%%%%%%%%%%%%%%%%%
\chapter{Enabling Analysis}
A comprehensive analytic system requires a variety of ways to manipulate and
looking at data. In this section we describe solutions for trend detection,
making comparisons and high level overview.
 

\section{Heatmap Perspective}
The heatmap widget has generic support for visualizing a time series data. It 
allows different time series to be interchanged within the heatmap itself, showing 
different perspectives. This was done to support the different types of
analytical tasks as we discussed in Chapter 3. These tasks, such as
finding seasonal trends of a component and finding the worst performing
component, are quite different, one requires a localized view showing how a
single component performed over time, the other requires a very high level view
designed to draw out outliers.

    % === Figure === 
	\begin{figure}
	 \centering  
	 \includegraphics[width=\columnwidth]{heatmap.png}
	 \caption[Heatmap Perspectives]{Showing the different heatmap perspectives. Left
	 displays the monthly perspective, centre displays the component perspective, and right displays the
	 global perspective}
	 \label{figure:heatmap}
	\end{figure}
	% ==============


Our visualization provides several different perspective views as seen in
Figure \ref{figure:heatmap}, all based on the occurrence score. The score of
each entity of each month is normalized by a divisor, which determines the type of semantic we want to show. The available
heatmap perspectives are listed below:

\begin{itemize} [noitemsep]
  \item Component-Max: An object component perspective where the score of each
  month is divided by the maximum component score over the selected time. This
  is the default perspective in the system.
  
  \item Month-Max: A monthly perspective where the score of each month is 
  divided by the maximum monthly importance score over all components
  in the selected time.
  
  \item Global-Max: A global perspective where the month score is divided by the
  maximum score of all components over the selected time. 
\end{itemize}
 
Here we illustrate how the perspective scoring works with an example of a time
over a period of three month: Suppose there are two entities, engine and brakes
and their occurrences scores over the three month are (1, 10, 100) and (10, 15,
20) respectively. In Table \ref{table:perspective} we show the entities' month
score under each perspective.

The monthly perspective draws out the
highest scored entity of each month, thus they are 10, 15 and 100 over the 3
months shown. Component view is localized for each entity, thus it is 100 for
engine entity and 20 for the brake entity over the 3 months. Finally global
perspective uses global maximum as the divisor, which is found in the engine
entity in the third month. 
 
    % === Table ===
    \begin{table}[h]
	%\begin{tabular}{| l | lll | lll | lll | lll | 
	\begin{tabular}{| l | lll | lll | lll | 
	      } 
	   % Column Heading
	   \hline
	   %& \multicolumn{3}{|c|}{Score} 
	   & \multicolumn{3}{|c|}{Month} 
	   & \multicolumn{3}{|c|}{Component} 
	   & \multicolumn{3}{|c|}{Global} \\
	   
	   % Data 
	   \hline
	   Engine & %1 & 10 & 100 &       % Original
	            1/10 & 10/15 & 100/100 &      % Month
	            1/100 & 10/100 & 100/100 &    % Component
	            1/100 & 10/100 & 100/100 \\   % Global
	            
	   Brake &  %10 & 15 & 20  &      % Original
	            10/10 & 15/15 & 20/100 &      % Month
	            10/20 & 15/20 & 20/20  &      % Component
	            10/100 & 15/100 & 20/100 \\   % Global
	   \hline
	\end{tabular} 
	\caption{Sample perspective based scores} 
	\label{table:perspective}
	\end{table}
	% ============
  
Each of the perspectives above answers different questions and has its own
advantages and disadvantages. The month-max perspective allows us to compare 
component-to-component by month, but comparison against adjacent cells are 
meaningless because each cell uses a different base value. The component-max 
perspective is the opposite, it allows us to see trends with a single entity, 
but it does not allow comparison across components. Lastly, the global-max 
perspective is good at showing the outliers and supports both month-to-month 
and component-to-component comparisons, but it is difficult to see overall
trends because the outliers, if any, will dominate and push all non-outliers
into the same scoring bin.

To put the different perspectives in better context, we list sample 
questions that can be answered with these different perspective views:
\begin{itemize} [noitemsep]
  \item Month-Max: In month X, which vehicle component had the most complaints?
  \item Component-Max: Are there more braking problems in the summer months or
  the winter month?
  \item Global-Max: What are the most unreliable vehicle components?
\end{itemize}

Going back to Figure \ref{figure:heatmap} as an example, one can some
interesting observations. From the component view in the centre, a person can
see that there are two distinct outliers in the second and third months of the
second year, in particular, one can see the scores getting lower, then there is
a resurgence around July and August in the second year. Switching to the monthly
perspective, one can observe that during the two year period, the most
significant components seem to alternate between the brake and engine component,
with the sole exception of accelerator appearing in a single month. Lastly, the
global perspective yields 3 outliers, February and March from the brake entity
and February from accelerator entity, however note the rest of the cells are
pushed into the lower brackets and not possible to detect any other trends.
 
The heatmap viewing perspective is at a global scope, thus a change in
perspective will affect all visible heatmaps. This keeps the interface
consistent and avoid viewers from switching to different modalities when they
shift their attention from one heatmap to another. The view switching mechanism
is realized as a drop-down control sitting atop the hierarchy filters and shows
the currently selected viewing mode.



\section{Comparison}
Comparison mode allows people to compare entity occurrences across
two different subsets of the data. To select data to compare, we provide
two sets of filter widgets which can be used to specify manufacturer,
make, model and model year. Each set of filters specifies a query,
which we will call Q1 and Q2, and each query is assigned a colour,
which is used in the visualization. For example, we can compare
Honda Civic (Q1) to Toyota Corolla (Q2), or we can compare Ford
Focus (Q1) against all other Ford vehicles (Q2), by not fully specifying Q2.

Two separate measures are used to render the comparison view.
The \emph{contribution sum is} the aggregated component score from the two
query sets: it reflects the overall importance of the component by emphasizing
the most frequently occurring components matching Q1 and
Q2. The \emph{percentage difference} describes the relative frequencies of a
component, whether it occurs more frequently under Q1 or Q2 relative
to the total contributions from Q1 and Q2 respectively. The percentage
score is calculated as the component score divided by the total
contribution. Then the percentage difference follows as percentage
score Q1 minus percentage score Q2, with the sign and magnitude indicating
which query set has the stronger presence of that component.

We made the decision to use percentage based comparisons because it
enables the comparison of query results of different sizes.
These scores are used to render the 3D view. Using the percentage
difference, the colour of the outline of a component indicates which
query set has the higher rate of complaints, and the opacity of the outline
indicates the strength of the difference. Using the contribution
sum, the standard hue and opacity encoding is used to indicate the
sum of the two query sets, giving an impression of the overall importance
of that component. Thus, a highly problematic component from
both queries will have a strong presence overall but with a faint outline,
while a lopsided but infrequently mentioned component will have
strong outline but barely visible interior colour.

% Comparison mode allows people to compare two different types of the same
% physical product specified by two different query sets Q1 and Q2. Every entity
% from a query set is matched against the same named entity in the other query
% set, producing an entity level based comparison. For example, we can compare how
% each car component fared against each other in a comparison of Honda Civic
% versus Toyota Corolla, or comparing Ford Focus against the industry form.
% 
% Two separate measures are used for the comparison view. A contribution sum and a
% percentage difference. The contribution sum is the aggregated entity score from
% the two query sets, it reflects the overall importance of the entity by
% emphasizing the frequently occurring entities in one or both query set. The
% percentage difference describes the relative frequencies of an entity, whether
% it occurs more frequently under Q1 or Q2 relative to the total contributions
% from Q1 and Q2 respectively. The percentage score is calculated as entity score
% divided by the total contribution. Then the percentage difference follows as
% percentage score Q1 minus percentage score Q2, with the sign and magnitude
% indicating  which query set has the stronger presence. We made the decision to
% use percentage based comparisons because it enables as to compare query results
% of different sizes.
% 
% We encode the sum score as the colour of the \threed component, based on the
% default colour scale. For the difference score, we render the measure as an
% outline around the object. The sign of the difference is encoded as one of two diverging
% colours, one for positive and one for negative values. The magnitude of the
% difference is encoded as the transparency of the outline, larger magnitude have
% more distinct, solid outlines compared to smaller values. Thus, a highly
% problematic object from both side will have a strong presence overall but with a
% faint outline, while a lopsided but infrequent problem will have strong outline
% but barely visible interior colour.

    % === Figure ===
	\begin{figure}
	 \centering  
	 \includegraphics[width=\columnwidth]{comparison.png}
	 \caption[Comparison View]{Left: Plymouth versus Chrysler, the brake appears to
	 be the dominant issue and Make 14 has the higher rate of complaints. Right:
	 Plymouth versus Jeep, the engine is the dominant issue and Make 1 has higher
	 rate of complaints.}
	 \label{figure:comparison}
	\end{figure}
	% ==============

As an example, see Figure \ref{figure:comparison}, where we compared Plymouth
against both the Jeep and Chrysler. At a glance most parts remain the same
with the exception of two outliers, Jeep appear to have a higher failure rate
than Plymouth, and Dodge has a higher failure rate in brakes component. The
overview could suggest that Plymouth is more reliable than both vehicles.
 
By default, comparison mode is turned off. It is activated when the viewer
switches the second hierarchy filter from the ``None'' position to a valid
selection. Subsequent query modifications are carried out in comparison mode
until the selection is turned to ``None'' again.

One limitation with this approach is our current usage of the total
contributions to calculate the percentage scores. Our total contribution is in
relations to the number of documents in the corpus, which may not be
the best indication of the overall contribution.

 
 
 
\section{Aggregation}
By default, the system treats each object individually rather than object
groups. For example ``seatbelt'', ``backrest'' and ``seat'' are all scored
separately, even though they are logically under the group ``seat''. This
setting allows people to isolate and identify unique problems accurately. There
are times, however, when this level of information is unnecessarily detailed and
a higher level of abstraction is desirable.

    % === Figure === 
	\begin{figure}
	 \centering  
	 \includegraphics[width=\columnwidth]{aggregation.png}
	 \caption[Aggregation View]{Top: Aggregation mode disabled. Bottom: Aggregation
	 mode enabled, note that the seat and engine now appears more prominent in the visualization.}
	 \label{figure:aggregation}
	\end{figure}
	% ==============
	
Aggregation mode mimics the type of high level rating system found on consumers
review websites. When aggregation mode is enabled, individual objects, and their
scores are aggregated up to the first level entities. In our specific case, the 
first level are the major sub-systems in a vehicle. Aggregated components
responds to interaction events as a single group, thus, selecting the
``seatbelt'' will select the entire ``seat'' subsystem.

%The visualization responds
%by making all child objects referencing the aggregated score of their parent
%subsystem. 

Figure \ref{figure:aggregation} shows a before and after illustration of using
aggregation mode. A default rendering is shown in the top portion, one can see
that brake is the most severe out of all components. The bottom shows the
aggregated view, one can observe that on a higher level, the engine and seat
subsystems are quite problematic.
 
Aggregation mode is enabled/disabled by a toggle switch located at the top
portion of the display interface. Aggregation mode works in
conjunction with comparison mode, allow people to make comparison of major
systems.
