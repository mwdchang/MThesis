\chapter{Scenarios and Study}

%%%%%%%%%%%%%%%%%%%%%%%%%%%%%%%%%%%%%%%%%%%%%%%%%%%%%%%%%%%%%%%%%%%%%%%%%%%%%%%%
\section{Scenarios}
The basic idea behind our visualization system is to present a visual overview
of the most salient objects in text document, without resorting to reading the
text documents one-by-one to get a summary of their contents. Here we present
several scenarios to illustrate our visualization system in action.

\subsection{Buying a Used Vehicle}
The family sedan has been in service for the past 10 years and is starting to
show its age. Michael, the owner and primary provider for his family of four had 
been thinking of buying a new sedan for the past three month and have been saving 
up for the big purchase. With the budgeting concerns more or less taken care of, 
Michael wants to make sure that the next family car is reliable and built by a 
manufacturer with a proven track record on safety standards. With these in mind, 
Michael decides that he does not want to buy the new models coming out this year, 
he believes it is too much risk buying new vehicles that has not been tested by 
drivers in real life. Michael starts off his search by talking to his friends and 
colleagues at work to get a sense of how they feel about certain manufacturers, 
and to get a sense of how their vehicles performed up-to-date. In his second step, 
Michael conducts research on various consumer product review websites to see how 
the expert rank the automotive manufacturers and their specific vehicles in terms 
of safety and reliability ratings, he also browsed open forums to see what type of 
problems can arise from current owners. Michael finds his search helpful but quite 
tedious because the information is difficult to digest; often the data is quite 
generic and provide no link to detailed information. After a week of searching, 
Michael finally picked out three potential models that he believes will meet 
his requirements, he contacted the used car dealerships to come in for a closer 
look in person.

The visualization system saves Michael the need to aggregate the data himself, which is tedious
due to the number documents available, and the number of things mentioned within
each document. Using our visualization system, it is possible for Michael to
speed up his research process because the system presents a visual summary. Using the time
filters Michael can narrow down the period of his research. He can then use the
hierarchy filters, as well as the comparison mode to see how manufacturers
perform relative to each other. Michael can use visual inspection to find
anything that are out of place, he can then use the lens widget to focus on the
the area of interest to see more details. Finally he can select the object into
focus and use the document widget to read detailed reports.


\subsection{Investigating Possible Problems}
Larry owns a 2009 Toyota Corolla; for the past two or three weeks while driving
on city streets he noticed a strange intermittent vibration coming from near the 
front-passenger side of the his car. Larry is mulling over whether to contact the 
dealership to have them do a checkup, he does not want to spend the time and
money to fix what might be a trivial issue, but at the same time he is worried that his 
might develop into a major problem which would cost more in the long run. He 
checked on the web to see if there are any other owners with the same issues, after 
browsing multiple forums and looking through all the owner complaints with similar 
characteristics he found about a dozen instances with half of them leading to more 
serious suspension issues. Larry decides that these provides more than enough evidence 
that the car may have a serious problem, he contacts the dealership to make 
appointment for a checkup.

The visualization system helps people explore items and issues that are not
necessarily known ahead of time. Instead of taking a guess at what the
issue is, Larry can use the lens widget tool and place the lens at the front of
the car.  The system will then display the available objects
covered by the lens and narrow his search results significantly. Larry can then
use the document widget and read the detailed complaints and decides if the
problem is serious enough to contact the dealership.

\subsection{Finding Causal and Related Issues}
Curly is an automobile defect investigator, he is in the process of gathering
information about a possible defect in the 2009 model of sedans. He leverage the
NHTSA database as a data source for consumer initiated complaint reports. He
enters the criteria one by one: manufacturer, model, make and the date range he
wants to search for. After reviewing several dozen reports with brake issues he
notices that there seem to be a trend emerging: over 50 percent of the reports
read so far also mentions problems with the accelerator pedal. Thinking that
there may be a connection between these two components Curly decides that he
should widen his search criteria to include both accelerator and brakes.

Our system suggest causal relations by explicitly highlighting the related
entities in text documents. Rather than having to account for related terms in
his head, Curly can select the brake component in the visualization, which will
high light all co-occurring entities, include the accelerator. A visual scan
will reveal the high occurring, which Curly can select, and use the document
panel to read the detailed complaint descriptions.


%%%%%%%%%%%%%%%%%%%%%%%%%%%%%%%%%%%%%%%%%%%%%%%%%%%%%%%%%%%%%%%%%%%%%%%%%%%%%%%%
\section{Study}
We created a user study to see how people can use our visualization to solve
specific problems. In particular we are interested to see how people respond to
using \threed interface for solving analytical tasks.

\subsection{Data and Task}
To remove any personal biases against particular makes of vehicles or
manufacturers. We perform a replacement scheme to remove known and common
identifiers and replace them with generic placeholders like ``Make1'',
``Model2'' \ldots etc. We tested our visualization with X different tasks as
seen in \ref{table:StudyTask}. Tasks 1 through 4 are objective tasks where we
would expect the same, or similar outcomes from our test trails. Tasks 5 through
7 are subjective tasks, they look at how the visualization is used to help with
decision making process.

\begin{table}[h]
\centering
\begin{tabular}{|l|l| p{4.0in}|}
\hline
 \# & Type & Task Description \\
\hline
\hline 
1 &  Objective & Select the component that has the highest rate of complaints in
year X (With heatmap feature disabled) \\
\hline
2 &  Objective & Select the component that has the highest rate of complaints in
year X with heatmap selection \\
\hline
3 &  Objective & Select the component(s) with the highest number of complaints\\
\hline 
4 &  Objective & User the lens widget to select region that contains the highest
number of issues\\
\hline
5 & Subjective & Identify the components that failed when component X also
failed \\
\hline
6 & Subjective & Given vehicle X and Y, which part in X contains more complaints
than the equivalent component in Y? Which part in vehicle Y contains more
complaints then the equivalent component in X? \\
\hline
7 & Subjective & Given a list of vehicles (X,Y,Z) of similar type and price
range, which vehicle would you purchase and why? \\
\hline
\end{tabular}\caption{Study Tasks}\label{table:StudyTask}
\end{table}
 
\subsection{Study Design}
The experiment is run one participant at a time. We begin the study by
describing the goal of the study and the nature of the dataset. We explain how
the colour scale is used to differentiate high occurring and low occurring
entities, and go over the widget functionalities. The next phase consist of warm
up tasks, our system prompt the participants to use specific features in the
correct manner, this gives the participants time to familiarize themselves with
the environment. After the tasks are completed, we give each participants X
minutes to explore the visualization on their own, we are on hand to answer any
questions they have at this time.

After this phase, we begin to task trials. We administer the tasks in order,
participants may take as much time as necessary to complete each trial. After
the trial completion we solicit additional feedback via a semi-formal interview,
in particular, we ask them what features they found to be useful, and whether
they encountered any issues with the visualization. Each experiment takes
approximately 60 to 90 minutes.

The experiment ran on a 60 inch display screen running at 1680 x 1050
resolution. The display is enhanced with a touch sensor that is capable of of
detecting up to X touch points. Video recording is used to record the
experiment.

\subsection{Study Result}
We recruited our participants from the graduate student population and the
computer science faculty.

\subsection{Discussion}