%%%%%%%%%%%%%%%%%%%%%%%%%%%%%%%%%%%%%%%%%%%%%%%%%%%%%%%%%%%%%%%%%%%%%%%%%%%%%%%%
% Appendix. For data files, study questions, full study results etc etc
%%%%%%%%%%%%%%%%%%%%%%%%%%%%%%%%%%%%%%%%%%%%%%%%%%%%%%%%%%%%%%%%%%%%%%%%%%%%%%%%
\appendix


%%%%%%%%%%%%%%%%%%%%%%%%%%%%%%%%%%%%%%%%%%%%%%%%%%%%%%%%%%%%%%%%%%%%%%%%%%%%%%%%
% Study questions
%%%%%%%%%%%%%%%%%%%%%%%%%%%%%%%%%%%%%%%%%%%%%%%%%%%%%%%%%%%%%%%%%%%%%%%%%%%%%%%%
\chapter{Study Procedure}

\noindent 
Warm up tasks, these are designed for the participants to become
familiar with the visualization and the system interactions.

\begin{itemize}[noitemsep]
  \item Select years 2000 and 2001 on the year slider.
  \item Select at least two components on the \threed vehicle model.
  \item Select a vehicle component using the lens' heatmap.
  \item Use the comparison function to compare two different vehicle
  manufacturers, then select a vehicle component from the \threed vehicle model.
\end{itemize}
 


\noindent 
Objective tasks, these tasks have specific requirements and answers, these tasks
are used to gauge the accuracy of participants perception of the \threed
visualization.

\begin{itemize}[noitemsep]
  \item Select the component with the highest rate of complaint overall in the
  year 1999.
  \item Select the component with the highest rate of complaint in July and
  August, from 2000 to 2001.
  \item Which manufacturer has the highest number of engine complaints in 1998,
  select this manufacturer from the manufacturer drop down
  \item Tell us, verbally, what other components in the vehicle are associated
  with complaints about windshield and wheel? Do not change the time or the
  hierarchy filters.
  \item Find a complaint that is associated with both engine and wheel
  components. Read it out when you find it. You can use whatever widgets  you
  want to accomplish this task.
\end{itemize}


\noindent 
Subjective tasks, these tasks are open-ended. There are used to see if, and how
participants use the visualization to solve analytical problems.

\begin{itemize}[noitemsep]
  \item Which manufacturer had the least complaints in January between 1997 and
  1998? What are these complaints about?
  \item Using the lens and the heatmap widgets, observe for any trends, patterns
  or outliers in the year 1997, tell us about your findings.
  \item Between 1997 and 2000, which of the following Make would you consider to
  purchase and why? MRF1:MAKE1 or MRF2:MAKE3? Assume they are similarly
  priced.
\end{itemize}

 
\noindent
Interview questions, conducted during a semi-structured interview. These
questions are designed to solicit subject feedback about the visualization and
design improvements.



\begin{itemize}
  \item Do you find the \threed visualization intuitive and easy to read? Why or
  why not?
  \begin{itemize}
    \item Do you think the visualization is useful for showing the vital safety
    issues?
  \end{itemize}
  
  \item What do you think of the lens widget? What do you like or not like about
  it?
  \begin{itemize}
    \item Do you understand the heatmap? Do you find it intuitive and do you
    believe it is easy to spot trends and outliers?
  \end{itemize}
  
  \item Which feature of the application do you most enjoy using? Which features
  do you find the last useful for looking at reliability issues? 
  
  \item Did you encounter any issues using any of the widgets? 
  \begin{itemize}
    \item  Prompt about possible usability issues and improvements.
  \end{itemize}
  
  \item Do you have any other suggestions or feedback about the application?
\end{itemize}