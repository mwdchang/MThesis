%%%%%%%%%%%%%%%%%%%%%%%%%%%%%%%%%%%%%%%%%%%%%%%%%%%%%%%%%%%%%%%%%%%%%%%%%%%%%%%%
% This covers the problem analysis portion of the thesis
%
% TODO:
% - Should probably explain more on tasks and requirements, tie back to
% motivations
%%%%%%%%%%%%%%%%%%%%%%%%%%%%%%%%%%%%%%%%%%%%%%%%%%%%%%%%%%%%%%%%%%%%%%%%%%%%%%%%
\chapter{Problem and Data Analysis}
\section{Problem Statement}
Within any text document which describes physical objects, the words may have 
two types of context. First, the context of the word in the sentence of the 
document (i.e., occurrence and co-occurrence relationships). Second, the context 
of the words in terms of its relation to the physical world, and its spatial 
relationship among other words in the document that describe physical objects. 
For example, the sentence ``Automatic door locks when used, will not
release from any of the four doors when engine is turned off.'' contains a
co-occurrence relationship among ``door'', ``door-lock'' and ``engine'',
however the physical context of these components in a real world automobile 
is quite different.

Traditional text visualization approaches such as tag clouds and word clouds falls 
into the first categorization above, they could be used to summarize text content 
by itself, or over a period of time through aggregation. However they lack the 
semantics to handle co-occurrence relations, and certainly the spatial and physical 
attributes in the text are presented out of context.  Other visualization approaches 
such as DocuBurst organize words spatially based on semantic of the ``IS-A'' 
relationship, while structural based visualizations such as WordTree(Suffix Tree) 
visualizes the most frequently occurring phrases or sentences. Thus far, we are not 
aware of any visualization which approaches text visualization by visualizing the 
real-world spatial context of the words in the text documents.

\section{Data Overview}
Each year thousands of reports are submitted to the NHTSA database and are made 
available to the public.  These reports consist of complaints from vehicle owners, 
reports from defect investigations and reports relating to manufacturer recalls. 
One of the tasks for buying a vehicle is to examine the safety and reliability 
of the vehicle. The NHTSA database offers a wealth of information to guide 
purchasing decisions, as well as inform insurance companies and automotive 
manufacturers about potentially serious safety and reliability concerns as reported 
through experiences of real drivers in realistic scenarios. However, we have yet 
to see any sophisticated ways of representing this dataset. Consumers have to
use conventional search forms that a large number of amounts of textual results; 
there are no mechanism to support concise overviews or dynamic details on demand. 
The step-by-step querying process also prevent  consumers from freely exploring 
the data, they need to have a preconceived notion of what they want to look for, 
thus likely prevent any type of unexpected discoveries.

On the other end of the spectrum, there exist consumer product website such as 
Consumer Reports and Edmunds. These website provide reviews and linear scale 
rating for different types of vehicles, but these ratings seldom provide details 
and in-depth analysis. In other words, the ratings are too coarsely grained. 
Yet another issue with these websites is that they are typically targeted at 
newer vehicle models, as such, it can be difficult to look up and compare 
ratings between new and old models.

\section{Tasks}
Our data consist of a collection of over 800K time-stamped complaint reports 
from 1995 till present day. While our data is temporal in nature and new 
reports are constantly being added, our system is not, strictly speaking, a 
real-time system that consumes streaming text data. Our system consume data in 
batches, and to our knowledge, there are no known external facing API available 
for monitoring updates. Having said that, our system share many of the same 
concern as real time text streaming applications. Rohrdanz \etal
\cite{ROH2011a} outline seven important tasks analytical tasks for working with text streams, 
including monitoring, decision making, change/trend detection, event tracking, 
historical retrieval, exploration and situational awareness. 

For the purpose of the vehicle defect reports, and for an audience of prospective 
car/used-car buyers, some of the vital tasks are:
\begin{itemize}
  \item Decision Making: ``Which vehicle should I buy?''
  \item Historical Retrieval: ``Are there any major concerns with vehicle X over
  the last 5 years?''
  \item Exploration: ``How does vehicle X compare to other vehicles in the same
  category?''
\end{itemize}

For a quality assurance engineer or investigator, the tasks may be:
\begin{itemize}
  \item Monitoring: ``Are there any new complaints relating to make Y?''
  \item Decision Making: ``Are there enough reports and evidences to warrant a
  full scale investigation or a recall?''
  \item Change and Trend Detection: ``For this type of vehicle, are the rate of
  complaints per month increasing or decreasing?''
  \item Situational Awareness: ``How does reports about my vehicle compare to
  the current state of the automobile industry''
\end{itemize}

In our design, we aim to create an application which aims to support these users
and tasks as they analyze these complaint reports. We take the view of the 
consumer as the primary stakeholder. Thus, our goal is to provide text-analytic 
visualization for helping consumers understand and explore vehicle safety issues, 
which in turn will affect their purchasing decisions

\section{Requirement Gathering}
To build a useful system, we need to understand the stakeholder�s mindset. We
start our initial requirement gathering by looking at websites dedicated to 
vehicle owners and potential car buyers, our sources include expert columns, 
question and answer forums, car-buying tips and product rating websites such 
as Consumer Reports and Edmunds. These resources gives us some insights to what 
the consumers are concerned about as well as deficiencies in the current vehicle 
buying research methods. The user forums give us an idea of what the consumers 
are concerned about, while the rating site and expert columns gives us some notion 
of what the consumers should be concerned about. Our findings revealed that, aside 
from the price factor, the next item people care about are safety and reliability; 
this makes sense, purchase of a vehicle is a big investment, the vehicle itself 
needs to be reliable to be used frequently and ensure the safety of its passengers. 
Basically, consumers want to know which brand/make they can trust. Other forum posts 
refer to existing problems, with the owners asking whether the problem is an 
isolated event or if the issues are wide spread, this indicates a need and willingness 
for detailed exploration. Car buying guides often advocate conducting thorough 
research on the vehicle and brand history, as well as leverage the experience of 
other owners. This is particularly important for used vehicles.

Based on these findings, we have four design requirement for our visualization
system:
\begin{itemize}
  \item R1: Provide an intuitive representation and make important items clearly
  visible;
  \item R2: Facilitate finding of trends, interesting facts and causal relations
  in the reports;
  \item R3: Allow multiple types of comparisons across different data
  dimensions;
  \item R4: Provide for reading of original complaint report text in the context
  of the visualization.
\end{itemize}

\section{Data Dimensions}
A typical complaint report consist of both structured and unstructured fields.
The structured fields related to the type of vehicle and other explicitly 
quantifiable variables relating to the incident (\eg, the odometer reading, 
number of injuries, the geo-location). The unstructured field is a free-form text, 
typically consist of several sentences describing the circumstances of the 
incident, as well as the parts that failed. 

\textbf{Hierarchy:} The organizational hierarchy of the vehicle manufacturers
(Manufacturer, Make, Model, Model Year) is an important aspect when purchasing 
a vehicle. Consumers not only want to compare model-to-model in terms of 
reliability and safety, they are also interested in whether the manufacturing 
companies are producing reliable vehicles in general and how manufacture rank 
with respect to each other. The organization hierarchy present a natural 
progression from the most general selection down to specific models, and thus 
supports successive refinement of queries. Note that the selection of 
organizational hierarchy as a data dimension is a domain specific decision, 
it does not generalize to text analytics for other domains.

\textbf{Spatial:} Many texts contain spatial variables that can be mapped to the
real, physical world. Sometimes these are explicitly stated, such as GPS 
coordinates which are mapped to a specific location on earth, others, such as 
an article about computer hardware, have implicit spatial locations in how the 
described components are arranged with respect to each other. Visualization 
offers a unique opportunity to gain insights about spatial patterns in a text 
document by showing inherent spatial relations that are not apparent from looking 
at the source text or their abstracted forms. For example, under Gestalt perception 
of proximity, highlighted objects can naturally form clusters and other patterns 
in \threed space, which can drive further analysis and exploration. Our spatial
variables are the vehicle components, for example the engine, wheel and doors. 
Like puzzle pieces, all together these component form a complete automobile.
\threed visualization is not a trivial task, due to perceptual limitations it
has many challenges when projected onto a \twod screen space
\cite{WAR2004b}. However, a carefully designed \threed interface can also create a simple,
compelling experience for people, particularly if the design goes above and beyond the 
constraints of normal reality \cite{Shneiderman2003}.
