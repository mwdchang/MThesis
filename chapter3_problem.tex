%%%%%%%%%%%%%%%%%%%%%%%%%%%%%%%%%%%%%%%%%%%%%%%%%%%%%%%%%%%%%%%%%%%%%%%%%%%%%%%%
% This covers the problem analysis portion of the thesis
%
% TODO:
% - Should probably explain more on tasks and requirements, tie back to % motivations
% - Show actual complaint report to better illustrate the data dimension section
% - Maybe picture of the NHTSA search interface
%%%%%%%%%%%%%%%%%%%%%%%%%%%%%%%%%%%%%%%%%%%%%%%%%%%%%%%%%%%%%%%%%%%%%%%%%%%%%%%%

\chapter{Problem and Data Analysis}
This chapter proposes our alternative approach for navigating a large body of
text. In particular, we introduce our dataset and problem domain: exploring 
vehicle complaint reports for reliability and safety analysis. We introduce our
goals and our requirement gathering process. Lastly we take a closer look at
our dataset records and specify the dimensions we want to visualize to support
our goals.

\section{Problem Statement}
When it comes to navigating, exploring and querying large bodies of text,
traditional visualization techniques often approach the problem from an abstract
perspective. These techniques explores the context of words in the sentence of
the documents, for example Word Tree \cite{Wattenberg2008} allows people to
explore the most frequently occurring sentence structures within a document.
Other types of visualization looks at summarizing the underlying text, popular 
visualizations on the web today such as tag-clouds and word-cloud emphasize
the most frequently occurring words or phrases, thus revealing possible themes in
the text. Still, other techniques looks at the language semantics, for example 
DocuBurst \cite{COL2009a} spatially organize words based on the ``IS-A'' relationship.

Looking at the underlying semantics helps us understanding the content and 
themes in the text documents. However, there is another type of word context 
that is not fully explored in the visualization community. Within any text 
document which describes physical objects, each word that describes a 
tangible object has relation to its physical, real world counterpart. The 
entities also have relation to each other in terms of their respective 
spatial positions. For example, the sentence ``Automatic door locks when 
used, will not release from any of the four doors when engine is turned off.'' 
contains not only a co-occurrence relationship among the entities ``door'' and 
``engine'', but also relates to where the doors and engine are located on a 
real world automobile.

Revealing these spatial relations of common words in text may enable new types of 
insights and exploration techniques. By mapping physical entities onto virtual
\threed representations it is possible to create a visualization environment
that resembles the real world, the environment can then be explored and queried with relative ease
due to existing familiarity of how these objects operates in real life. There are 
many types of text that carries these sort of physically mappable vocabulary: 
product reviews, technical manuals, maintenance logs, and our dataset: vehicle 
defect reports.

So far, we are not aware of any exploratory visualizations which approach text
visualization by visualizing the real-world spatial context of the words in
text. Thus, our work looks at revealing these real world relations in a manner
that is useful for conducting text-analytic activities.


\section{Data Overview}
Every year thousands of reports are submitted to the NHTSA database and are made 
available to the public. These reports consist of complaints from vehicle owners, 
reports from defect investigations and reports relating to manufacturer recalls. 
In total, our data consist of a collection of over 800K+ time-stamped complaint 
reports dating from 1995 to present day.

One of the tasks for buying a vehicle is to examine the safety and reliability 
of the vehicle. The NHTSA database offers a wealth of information to guide 
purchasing decisions, as well as inform insurance companies and automotive 
manufacturers about potentially serious safety and reliability concerns as reported 
through experiences of real drivers in realistic scenarios. However, we have yet 
to see any sophisticated ways of representing this dataset. Consumers have to
use conventional search forms that returns a large number of amounts of textual
results; there are no mechanism to support concise overviews or dynamic details on demand. 
The step-by-step querying process also prevent  consumers from freely exploring 
the data, they need to have a preconceived notion of what they want to look for, 
thus likely prevent any type of unexpected discoveries.

On the other end of the spectrum, there exist consumer product website such as 
Consumer Reports\footnote{http://www.consumerreports.org/} and
Edmunds\footnote{http://www.Edmunds.com}. These website provide reviews and
linear scale ratings for different types of vehicles, but these ratings seldom
provide details and in-depth analysis. In other words, the ratings are too
coarsely grained. Yet another issue with these websites is that they are
typically targeted at newer vehicle models, as such, it can be difficult to look
up and compare ratings between new and old models.
 
\section{Tasks}
While our data is temporal in nature and new 
reports are constantly being added, our system is not, strictly speaking, a 
real-time system that consumes streaming text data. Our system consume data in 
batches, and to our knowledge, there are no known external facing API available 
for monitoring updates. Having said that, our system share many of the same 
concern as real time text streaming applications. Rohrdanz \etal
\cite{ROH2011a} outline seven important tasks analytical tasks for working with
text streams which we found to be suitable for our problem domain. These
includes monitoring, decision making, change/trend detection, event tracking,
historical retrieval, exploration and situational awareness.

For the purpose of analyzing vehicle defect reports, and for an audience of prospective 
car/used-car buyers, some of the vital tasks are:
\begin{itemize}[noitemsep]
  \item Decision Making: ``Which vehicle should I buy?''
  \item Historical Retrieval: ``Are there any major concerns with vehicle X over
  the last 5 years?''
  \item Exploration: ``How does vehicle X compare to other vehicles in the same
  category?''
\end{itemize}

From the perspective of an assurance engineer or incident investigator, the
tasks may be:
\begin{itemize}[noitemsep]
  \item Monitoring: ``Are there any new complaints relating to vehicles of make
  Y?''
  \item Decision Making: ``Are there enough reports and evidences to warrant a
  full scale investigation or a recall?''
  \item Change and Trend Detection: ``For this type of vehicle, are the rate of
  complaints per month increasing or decreasing?''
  \item Situational Awareness: ``How does reports about my vehicle compare to
  the current state of the automobile industry''
\end{itemize}

In our design, we aim to create an application which aims to support these users
and user tasks as they analyze these complaint reports. We take the view of the 
consumer as the primary stakeholder. Thus, our goal is to provide text-analytic 
visualization for helping consumers understand and explore vehicle safety issues, 
which in turn will affect their purchasing decisions

\section{Requirements}
To build a useful system, we need to understand the stakeholder�s mindset. We
start our initial requirement gathering by looking at websites dedicated to 
vehicle owners and potential car buyers, our sources include expert columns, 
question and answer forums, car-buying tips and product rating websites such 
as Consumer Reports and Edmunds. These resources give us some insights to what 
the consumers are concerned about, as well as any deficiencies and usability issues 
in the current research method. Using both the forums and expert websites give us dual
perspectives: the forums tell us what the consumers are thinking, while the expert 
websites tell us what the consumers ought to think.

Our findings revealed that, aside 
from the price factor, the next item people care about are safety and reliability. 
This makes sense, purchase of a vehicle is a big investment, the vehicle itself 
needs to be reliable to be used frequently and ensure the safety of its passengers. 
In general, consumers want to know which brand/make they can trust. Other forum
posts refer to existing problems, with the owners asking whether the problem is an 
isolated event or if the issue is widely spread, this indicates a need and
willingness for detailed exploration. Car buying guides often advocate conducting thorough 
research on the vehicle and brand history, as well as leverage the experience of 
other owners. This is particularly important for used vehicles.

Based on these findings, we propose the following four design requirements for
our visualization prototype:
\begin{itemize}[noitemsep]
  \item R1: Provide an intuitive representation and make important items clearly
  visible;
  \item R2: Facilitate finding of trends, interesting facts and causal relations
  in the reports;
  \item R3: Allow multiple types of comparisons across different data
  dimensions;
  \item R4: Provide for reading of original complaint report text in the context
  of the visualization.
\end{itemize}

\section{Data Dimensions}
A typical complaint report consist of both structured and unstructured fields.
The structured fields related to the type of vehicle and other explicitly 
quantifiable variables relating to the incident (\eg, the odometer reading, 
number of injuries, the incident location). The unstructured field is a
free-form text field, it typically consist of several sentences describing the
circumstances of the incident.

\textbf{Spatial:} Many texts contain spatial variables that can be mapped to the
real, physical world. Sometimes these are explicitly stated, such as GPS 
coordinates which are mapped to a specific location on earth, others, such as 
an article about computer hardware, have implicit spatial locations in how the 
described components are arranged with respect to each other. Visualization 
offers a unique opportunity to gain insights about spatial patterns in a text 
document by showing inherent spatial relations that are not apparent from looking 
at the source text or their abstracted forms. For example, under Gestalt perception 
of proximity, highlighted objects can naturally form clusters and other patterns 
in \threed space, which can drive further analysis and exploration. Our spatial
variables are the vehicle components, for example the engine, wheel and doors. 
Like puzzle pieces, all together these component form a complete automobile.
\threed visualization is not a trivial task, due to perceptual limitations it
has many challenges when projected onto a \twod screen space
\cite{WAR2004b}. However, a carefully designed \threed interface can also create a simple,
compelling experience for people, particularly if the design goes above and beyond the 
constraints of normal reality \cite{Shneiderman2003}.

\textbf{Time:} Time is almost omnipresent when dealing with streaming or reporting data, 
whether it is the creation date or the timestamps within the document itself. 
For our specific dataset, we suspect that much interesting information can be 
gained by supporting high level activities such as seeing how a particular 
object changes over time, and comparison of season to season statistics. Being
able to search, compare and extrapolate trend information across time is vital for 
understanding the underlying data, thus making time an obvious choice for our 
visualization.

\textbf{Hierarchy:} The organizational hierarchy of the vehicle manufacturers
(Manufacturer, Make, Model, Model Year) is an important aspect when purchasing 
a vehicle. Consumers not only want to compare model-to-model in terms of 
reliability and safety, they are also interested in whether the manufacturing 
companies are producing reliable vehicles in general and how manufacture rank 
with respect to each other. The organization hierarchy present a natural 
progression from the most general selection down to specific models, and thus 
supports successive refinement of queries. Note that the selection of 
organizational hierarchy as a data dimension is a problem specific decision, 
it does not generalize to text analytics for other domains.


